\section[Conclusion]{Conclusion\footnote{Stephan}}
% https://www.wikihow.com/Write-a-Conclusion-for-a-Research-Paper
% 1. Restate the topic. You should briefly restate the topic as well as explaining why it is important.
% 2. Restate your thesis. Aside from the topic, you should also restate or rephrase your thesis statement.
% 3. Briefly summarize your main points. Essentially, you need to remind your reader what you told them in the body of the paper.
% ( 4. Add the points up. If your paper proceeds in an inductive manner and you have not fully explained the significance of your points yet, you need to do so in your conclusion. )
% ( 5. Make a call to action when appropriate. If and when needed, you can state to your readers that there is a need for further research on your paper's topic. )
% 6. Answer the “so what” question.

% Phrases: ...A key limitation of the actual setup is that
The developed localization solution was for the pipeless plant, a prototype of a chemical production plant which has a size of 3 by 4 meter. In this plant the vessel will be transported by AGVs from one station to another. In the actual setup only a camera, which is installed above the plant, was used to detected the AGVs and estimate their positions. The problem with this technology is the bad detection of the LED pattern from the AGVs during bright light conditions and also the space limitation. Another big disadvantage was the big computation effort which made the system also very slow. The main task of this project was now to find an alternative tracking solution. After the decision for a RFID system with passive tags as landmarks a prototype was build to verify the good solutions from the simulations. The implemented procedure was an initialization algorithm in which the the AGV makes an 360$^\circ$ turn and estimates its position and its orientation based on measurements during these turn. This project has shown that it is possible to put a reader on an AGV and detect passive tags with a reader and an antenna in an detection area of 14 cm. It has also been found out that there is inconsistent relation between the received signal strength (RSSI) of the detected tags and the distance between the tag and the reader. This leads to the fact that the estimation of the distance based on the RSSI is not trivial and was only solvable in a very simple and inaccurate way during the project. Based on the results from the initialization procedure, it can be concluded that it possible to estimate the position of the AGV with an average accuracy of around 2.5 cm and an estimation error of the orientation of around 23$^\circ$. This is not a very good results and fulfils only the minimal requirements. This project has also clearly shown that the anti-collision algorithm of the reader gives  very inefficient results and leads to an unknown amount of time until all tags in the detection area are identified. It was not possible to develop a running and faultless system, but is has been demonstrated that the system is light independent and space unlimited. Summing up all results, it can be concluded that the system is a promising technology for the indoor localizing on the small plant and gets rid of a lot of the current disadvantages. 