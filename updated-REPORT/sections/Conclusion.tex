\section{Conclusion}\label{Sec_Conc}
% https://www.wikihow.com/Write-a-Conclusion-for-a-Research-Paper
% 1. Restate the topic. You should briefly restate the topic as well as explaining why it is important.
% 2. Restate your thesis. Aside from the topic, you should also restate or rephrase your thesis statement.
% 3. Briefly summarize your main points. Essentially, you need to remind your reader what you told them in the body of the paper.
% ( 4. Add the points up. If your paper proceeds in an inductive manner and you have not fully explained the significance of your points yet, you need to do so in your conclusion. )
% ( 5. Make a call to action when appropriate. If and when needed, you can state to your readers that there is a need for further research on your paper's topic. )
% 6. Answer the “so what” question.

% Phrases: ...A key limitation of the actual setup is that
The developed localization solution was for the pipeless plant, a prototype of a chemical production plant which has a size of 3 by 4 meter. In this plant the vessel will be transported by AGVs from one station to another. In the actual setup only a camera, which is installed above the plant, was used to detected the AGVs and estimate their positions. The problem with this technology is the bad detection of the LED pattern from the AGVs during bright light conditions and also the space limitation. Another big disadvantage was the big computation effort which made the system also very slow. The main task of this project was to find an alternative tracking solution. During the project group phase differnt localization technologies were evaluated. With respect to the outcoming reseaches about triangulation, map-based-localizaion, pattern recognizion and localizaion via radio frequency identification the last RFID based localization of the AGV with passive tags as landmarks turned out to be the most promising among those four. With information of a similar project realized by the FH Dortmund a model to evaluate sample data and a localization algorithm was created in Matlab. This results of the simulation were promising and  therefore used during the decision making process about the actual hardware setup. With an demonstration board with the size of 30 cm x 30 cm the initialization procedure algorithm was implemented in which the AGV performes and 360$^\circ$ turn and estimates its position and its orientation based on measuremtns during this movement. With respect to this solutions it can be said that it is possible to assemble a reader on an AGV and detect passive tags with its antenna in a range of 14 cm. It also has been found out that an inconsistent realation between the revieved signal strength (RSSI) of the detected TAGs and the distance based on the RSSI is not generally trivial and was only solved in a rather simple and unriliable way during the project. Based on the results computed by the initializaion prcedure, it can be concluded that it is possible to estimate the position of the AGV with an average accuracy of aroung 2.5 cm and an estimation error of the orientation of around 23$^\circ$. Compared to the former localization set up this solutions, especially with respect to the orientation error, are not perfelty satisfying and just minimal requirements are fullfilled. The recieved data from the RFID reader have furthermore clearly shown that the anti-collision algorithm used by the reader leads to an unknown amount of time until each and every TAG in the detection area is identified. Summed up a model based demonstrator was realized which on the one hand does not improve the accuracy of the localizaion of the plant under good light conditions especially with respect to the orientation but on the other hand a promising technology for indoor localizaion with light independedcy, respectivelay cheap costs and highly scalability was found. 




%After the decision for a RFID system with passive tags as landmarks a prototype was build to verify the good solutions from the simulations. The implemented procedure was an initialization algorithm in which the the AGV makes an 360$^\circ$ turn and estimates its position and its orientation based on measurements during these turn. This project has shown that it is possible to put a reader on an AGV and detect passive tags with a reader and an antenna in an detection area of 14 cm. It has also been found out that there is inconsistent relation between the received signal strength (RSSI) of the detected tags and the distance between the tag and the reader. This leads to the fact that the estimation of the distance based on the RSSI is not trivial and was only solvable in a very simple and inaccurate way during the project. Based on the results from the initialization procedure, it can be concluded that it possible to estimate the position of the AGV with an average accuracy of around 2.5 cm and an estimation error of the orientation of around 23$^\circ$. This is not a very good results and fulfils only the minimal requirements. This project has also clearly shown that the anti-collision algorithm of the reader gives  very inefficient results and leads to an unknown amount of time until all tags in the detection area are identified. It was not possible to develop a running and faultless system, but is has been demonstrated that the system is light independent and space unlimited. Summing up all results, it can be concluded that the system is a promising technology for the indoor localizing on the small plant and gets rid of a lot of the current disadvantages. 