\section{Future Work}\label{Sec_fut}
After a proof-of-concept for an RFID based localization system has been built and a first demonstration set-up has been built the disadvantages and limitations of the prototype were evaluated. According to these results several points of improvement and extension were found and categorized into a hardware and a software section. 
\subsection{Hardware}
\begin{itemize}
\item The AGVs are feed by an included 12V battery which provides the power for all included electronical devices. This 12V power supply is available on board and is suggested to be used. Currently the WiFi-Module and the RFID-Reader are fed by an external powerbank since a 5V power supply is needed. In terms of one zentralized power supply a 12 V to 5 V converter can be installed and connected to the reader and wifi module.
\item As a first setup a demonstration area of 3 x 3 tags was build. In this rather small area the initializaion procedure was developed, but a real time localization while a path is followed by an AGV was not possible since the 30cm x 30cm was simply to small. For futrue research in terms of localizaion on a specified path additional TAGs can be included to the area of operation. Since the RFID concept is highly scalable the only change that needs to be made in the algorithm is the insertion of the additional TAG into the lookup table.
\item Currently the Robot No. 1 is the only AGV which is equipped with the RFID technology. To run the plant with multible AGVs the remaining robots needs to be upgraded.\\
\end{itemize}
\subsection{Software}
\begin{itemize}
\item During the initalizatin procedure a 360$^\circ$ turn is performed. The desired turn around 45$^\circ$ is realized by a driving time of 1125 ms. But it needs to be said that this movement is highly dependend on disturbances like changing battery charge and plant underground. For the future developers it is suggested to use the encoders of the robot wheels as a determination of the orientation instead of the parameter time.
\item As an alternative localization technology was found several code lines in the current code can be deleted since the camera and image processing is simply not used anymore. With a clean code an improvement of processing time will be achieved.
\item As a last point it can be said that even though a localization with RFID is now possible the results are not 100 percent realiable and the accuracy especially with respect to the orientation is not satisfying so far. As an improvement the triangulation algorithm has to be optimized and or a second RFID-antenna has to be added under the AGV to reduce measurment errors.
\end{itemize}