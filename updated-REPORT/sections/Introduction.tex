\section{Introduction}

In today\textquoteright s world, the development and implementation of the positioning system for the autonomous vehicle in a confined space remains to be a major issue and hindrance to a better control system. Though there exists many types of local positioning systems, the precision remains to be still a challenge. This problem becomes critical in a place of no GPS access.
This project aims at investigating various methods of indoor localization and to develop a proof-of-concept for the existing pipeless plant setup. \\

In the past years students and researchers at the Process Dynamics and Operations group at the TU Dortmund have developed this plant with vision based positioning system which need to be replaced to improve the overall efficiency of the system. Both the old and newly implemented techniques are written in C\# that sends the position update to the Python based controller code.\\

In this project, various techniques were discussed and ``\textit{Radio Frequency Identification}'' was chosen to be the ideal technique which would be further discussed in section \ref{Sec_selectionp} and \ref{Sec_theor}. The hardware implementation is explained in section \ref{Sec_Imp}. Results of the simulation and experiment is discussed in section \ref{Sec_Sim} and \ref{Sec_Imp}.\\

 Based on the experiment, conclusion and future work are given in section \ref{Sec_Conc} and \ref{Sec_fut} respectively.






