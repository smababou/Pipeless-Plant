\section{Introduction}

In today\textquoteright s world, the development and implementation of the positioning system for the autonomous vehicle in a confined space remains to be a major issue and hindrance to a better control system. Though there exists many types of local positioning systems, the precision remains to be still a challenge. This problem becomes critical in a place of no GPS access.
This project aims at investigating various methods of indoor localization and to develop a proof-of-concept for the existing pipeless plant setup. \\

In the past years students and researchers at the Process Dynamics and Operations group at the TU Dortmund have developed this plant with vision based positioning system which need to be replaced to improve the overall efficiency of the system. Both the old and newly implemented techniques are written in C\# that sends the position update to the Python based controller code.\\

In this project, various potential positioning techniques were discussed and their pros and cons were compared. The different localization methods would be further discussed in section \ref{Sec_selectionp}.After a thorough four different alternatives including a triangulation based methods for indoor applications, another pattern recognition based method (such as QR-codes), map-based localization, \textit{Radio Frequency Identification} was chosen to be the ideal technique. It is a versatile technology with multiple application areas, e.g. access control, race tracking and positioning. Automated multi-agent systems are increasingly utilizing RFID for localization as the technology has been proven to have many advantages over vision based positioning systems \ref{Sec_theor}.There are two potential ways to implement an RFID localization system namely active and passive. The latter is based on comparatively many passive tags, uniformly placed, on the ground of the plant area and active readers on the AGVs. The latter option was chosen for the project based on cost efficiency, system scalability and from literature proven applicability. \\

An RFID system is made up of two parts: a tag and a reader. RFID tags are embedded with a transmitter and a receiver. The RFID component on the tags has two parts: a microchip that stores and processes information, and an antenna to receive and transmit a signal, which partly contains the unique ID of the tag. The hardware components which are added to the AGVs comprise an RFID antenna, an RFID reader and a WiFi module. Further hardware implementation is explained in section \ref{Sec_Imp}. The WiFi module then transmits the reader data through the local network, using TCP/IP communication. The system data is lastly received by a PC that represents the control hub of the plant through a framework implemented in C Sharp (\texttt{C\#}). A TCP-Client was established in the \texttt{C\#} framework in order to handle incoming RFID data. The WiFi-module continuously sends data and the algorithm calculating AGV position prompts for this data as position and/or orientation of an AGV needs to be computed.\\

The implemented positioning algorithm requires the ID and the received signal strength indication (RSSI) of at least three RFID tags to calculate the position of the antenna. The RSSI gives a relation between the detected tag and the distance to it, in other words, a radius. The system has a record of the position of each tag and the ID of each tag hence holds information about the uniquely defined position of the tag. With three positions and the three corresponding radii, one can use trilateration to compute the position of the antenna. This concept was developed into simulation which would be discussed in section \ref{Sec_Sim}. The experiment bbbased results will be explained in \ref{Sec_Imp}. Based on the experiment, conclusion and future work are given in section \ref{Sec_Conc} and \ref{Sec_fut} respectively.





