\section*{Abstract} %

The pipeless plant at the Process Dynamics and Operations group is an experimental setup of Automated Guided Vehicles (AGVs) moving between various stations. The AGVs dynamically change trajectories in an operational mode based on a Model Predictive Control (MPC) scheme with the objective to get from one station to the other while at the same time avoiding each other. The current positioning system is based on pattern recognition where the system tracks each AGV based on a unique pattern of LEDs via a camera that overlooks the plant. The vision-based positioning system displays some flaws and should be replaced by a system more adapted to the actual operational environment of the experimental plant. \\

The project aimed at first evaluating different potential positioning systems, selecting one of them based on defined metrics. A proof-of-concept was developed based on the chosen technology for the experimental pipeless production plant in a model-driven fashion and the 
Radio-Frequency Identification (RFID) was chosen for further evaluation and implementation.
The system detects RFID tags with an reader and an antenna. The reader receives the unique Identification (ID)  and the RSSI (Received signal strength indication) of the tag which is being used for calculating the position of the robot. \\

For the estimation of the position part, the WiFi module then transmits the reader data through the local network, using TCP/IP communication. The system data is lastly received by a PC that represents the control hub of the plant through a framework implemented in \texttt{C\#}. The algorithm which calculates the position of the AGV prompts for this data as position and/or orientation of an AGV that needs to be computed. 

